\chapter{Инфраструктурно-экономическая часть}
\section{Поддержка расширений}
В Ярославской лаборатории FRUCT Redmine установлен на сервер c операционной
системой Debian с помощью системы управления пакетами, через которую также
происходит обновление. Особенностью обновлений является то, что система
управления пакетами полностью перезаписывает всё содержимое каталога с Redmine,
а значит при каждом обновлении необходимо заново устанавливать все расширения.
В настоящий момент в Redmine установлено 7 плагинов и 15 патчей, что делает
процесс обновления трудозатратным. Следовательно необходимо автоматизировать
процесс установки расширений.
\inspicture{redmine-update}{Процесс обновления Redmine}{0.6}
Следующие проблемы при обновлений возникает из-за несовместимости плагинов и
патчей с новой версией приложения:
\begin{itemize}
  \item патч не применяется из-за изменения исходных кодов;
  \item плагин некорректно работает.
\end{itemize}
На рисунке \ref{picture:redmine-update} приведены шаги, включающие процесс
обновления. Как видно из рисунка, следствием обновления приложения будет
внесение изменений в плагины и патчи. Необходимо внедрить инструменты
позволяющие вести разработку и поддержку данных расширений.

\subsection{Mercurial Queues}
Mercurial Queues (MQ) "--- это расширение для системы контроля версий
Mercurial, которое позволяет автоматизировать операции, связанные с поддержкой
патчей. Как показано на рисунке \ref{picture:mq-repo}, MQ хранит патчи в
репозитории, расположенном внутри метаданных родительского репозитория. MQ
хранит список всех патчей в файле \textit{series} (часть файла опущена):
\small{\begin{lstlisting}
wiki_external_filter_plugin.patch
wiki_formatting_routine.patch
fruct_logo.patch
\end{lstlisting}}
\inspicture{mq-repo}{Устройство Mercurial Queues}{0.4}
Работа с патчами в MQ организована в виде стеке и в данном файле определён
порядок помещения файлов в стек. Команда \textit{hg qpush} помещает патч на
вершину стека и таким образом применяет его, а точнее, записывает патч как
коммит в родительский репозиторий, см. рисунок \ref{picture:mq-repo}. Команда
\textit{hg qpop} снимает патч, находящийся на вершине стека. MQ отслеживает
применённые патчи, то есть текущее состояние стека, и хранит их список в файле
\textit{status}:
\small{\begin{lstlisting}
937b07ba19ea9d98c9e3433e5ba535bc64ddecaf:wiki_external_filter_plugin.patch
7c32cb90bb7f23cfdb9d60a130cc3a11ef426814:wiki_formatting_routine.patch
\end{lstlisting}}
Каждый коммит характеризуется названием, расположенным после двоеточия, и
идентификатором связанного с ним коммита в родительском репозитории.
Расширение MQ предоставляет множество команд для управления патчами, основные
из них:
\begin{itemize}
  \item \textit{qnew} создание патча;
  \item \textit{qpop} снятие патча;
  \item \textit{qpush} применение патча;
  \item \textit{qdiff} просмотр изменений, вносимых патчем;
  \item \textit{qrefresh} сохранение изменений в патч;
  \item \textit{qcommit} сохранение изменений в репозиторий патчей;
  \item \textit{qfinish} оформить патч, как коммит.
\end{itemize}

\subsection{Управление патчами при помощи MQ}
Для того, чтобы использовать MQ был создан репозиторий внутри
установочной директории Redmine. Метаданные репозитория хранятся в папке
\textit{.hg} и не затрагиваются обновлением приложения. В репозиторий
были добавлены файлы, к которым будут применятся патчи. Затем был
инициализирован MQ репозиторий командой \textit{hq init --mq}.

Все патчи были занесены MQ репозиторий и плагины были также оформлены в виде
патчей для того, чтобы автоматизировать процесс их установки.
Автоматизированный процесс обновления приложения показан на рисунке
\ref{picture:mq-workflow}. Сначала снимаются все патчи, с помощью команды
\textit{hg qpop -a}, затем приложение обновляется и все патчи применяются с
помощью команды \textit{hg qpush -a}. С помощью данного подхода удалось
избежать ручной установки плагинов и применения патчей с помощью утилиты
\textit{patch}.

При обновлении Redmine некоторые расширения могут стать неработоспособными и
потребовать внесения изменений. Был внедрён процесс, показанный на рисунке
\ref{picture:mq-workflow}, позволяющий эффективно осуществлять поддержку
расширений при обновлениях. Далее будет описан данный процесс. Допустим, что
при обновлении перестал работать один из патчей. Необходимо извлечь указанный
патч на вершину стека с помощью команды \textit{hg qpop <имя\_патча>}. Внести
изменения в исходные коды приложения и по окончании записать изменения в файл
патча с помощью команды \textit{hg qrefresh}. Поскольку патчи хранятся в MQ
репозитории, то следующим шагом следует выполнить коммит командой \textit{hg
qcommit}, чтобы зафиксировать изменения.

\inspicture{mq-workflow}{Работа с Mercurial Queues}{1}

Инструмент MQ был успешно внедрён в процесс разработки расширений, что помогло
снизить трудозатраты на поддержку расширений при обновлениях Redmine.

\section{Метрики расширений}
\section{Отчёт по возврату наработок сообществу}


%%% Local Variables: 
%%% mode: latex
%%% TeX-PDF-mode: t
%%% TeX-master: "diploma"
%%% End: 
