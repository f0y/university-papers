\chapter{Инфраструктурно-экономическая часть}
\section{Обеспечение поддержки при обновлениях Redmine}
В Ярославской лаборатории FRUCT Redmine установлен на сервер на котором
инсталиирована опреационная система Debian. Redmine был установлен с помощью
системы управления пакетами из репозитория и обновляется также с помощью неё.
В главе \ref{section:implementation} был разработан набор расширений, которые
следует установить в Redmine. Плагины могут быть установлены 
копированием в папку vendor/plugins относительно директории, в которую
установлен Redmine. А патчи устанавливаются с помощью утилиты patch. Установка
данных расширений не представляет проблемы, если Redmine не планируется
обновлять. При обновлении Redmine возможны следующие проблемы:
\begin{itemize}
  \item патч не применяется, из-за изменения исходных кодов;
  \item плагин перестал работать с новой версией приложения;
\end{itemize}
Также если Redmine установлен из репозитория, то при обновлении происходит
замена исходного кода и следовательно все применёные патчи снимаются. Таким
образом процесс обновления Redmine выглядит, как показано на рисунке
\ref{picture:redmine-update}.

\inspicture{redmine-update}{Процесс обновления Redmine}{0.6}

Сложность данного процесса быстро растёт с увеличением количества установленных
патчей и плагинов, а также сильно зависит от частоты с которой происходит
обновление системы. В настоящий момент в Redmine установлено 7 плагинов и 15
патчей, что делает процесс обновления трудозатратным.

Mercurial Queues (MQ) "--- это инструмент встроенный в систему контроля версий
Mercurial, который позволяет автоматизировать операции, связанные с поддержкой
патчей. Основные возможности MQ:
\begin{itemize}
  \item применение и снятие патчей;
  \item хранение патчей в репозитории;
  \item списки фильтрации.
\end{itemize}
Работа с патчами в MQ организована в виде стека. Помещение патча в стек
производит его применение, а извлечение патча из стека его снятие. Расширение
MQ предоставляет несколько команд для управления патчами:
\begin{itemize}
  \item qnew создание патча;
  \item qpop снятие патча;
  \item qpush применение патча;
  \item qdiff просмотр изменений вносимых патчем;
  \item qrefresh сохранение изменений в патч;
  \item qcommit сохранение изменений в репозиторий.
\end{itemize}


На рисунке \ref{picture:mq-workflow} показана основные сценарии работы с MQ. 
\inspicture{mq-workflow}{Работа с Mercurial Queues}{1}

Внедрение MQ позволило уменьшить трудозатраты, связанные с поддержкой патчей
при обновлении приложения. 

\section{Метрики расширений}
\section{Отчёт по возврату наработок сообществу}


%%% Local Variables: 
%%% mode: latex
%%% TeX-PDF-mode: t
%%% TeX-master: "diploma"
%%% End: 
