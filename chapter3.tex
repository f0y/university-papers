\chapter{Поддержка расширений}
\section{Внедрение Mercurial Queues}
В Ярославской лаборатории FRUCT, Redmine установлен на сервер c операционной
системой Debian GNU/Linux. Установка была произведена с помощью системы
управления пакетами, через которую также происходит обновление. Особенностью обновлений
является то, что система управления пакетами полностью перезаписывает всё
содержимое каталога с Redmine, а значит при каждом обновлении необходимо заново
устанавливать все расширения. В настоящий момент в Redmine установлено 7
плагинов и 15 патчей, что делает процесс обновления трудозатратным.
Следовательно необходимо автоматизировать процесс установки расширений.
\inspicture{redmine-update}{Процесс обновления Redmine}{0.6}
Следующие проблемы при обновлений возникает из-за несовместимости плагинов и
патчей с новой версией приложения:
\begin{itemize}
  \item патч не применяется из-за изменения исходных кодов;
  \item плагин некорректно работает.
\end{itemize}
На рисунке \ref{picture:redmine-update} приведены шаги, включающие процесс
обновления. Как видно из рисунка, следствием обновления приложения будет
внесение изменений в плагины и патчи. Необходимо внедрить инструменты
позволяющие вести разработку и поддержку данных расширений.

\subsection{Mercurial Queues}
Mercurial Queues (MQ) \cite{mq} "--- это расширение для системы контроля версий
Mercurial, которое позволяет автоматизировать операции, связанные с поддержкой
патчей. Как показано на рисунке \ref{picture:mq-repo}, MQ хранит патчи в
репозитории, расположенном внутри метаданных родительского репозитория. MQ
хранит список всех патчей в файле \textit{series} (часть файла опущена):
\small{\begin{lstlisting}
wiki_external_filter_plugin.patch
wiki_formatting_routine.patch
fruct_logo.patch
\end{lstlisting}}
\inspicture{mq-repo}{Устройство Mercurial Queues}{0.4}
Работа с патчами в MQ организована в виде стеке и в файле \textit{series}
определён порядок помещения файлов в стек. Команда \textit{hg qpush} помещает
патч на вершину стека и таким образом применяет его, а точнее, записывает патч
как коммит в родительский репозиторий, см. рисунок \ref{picture:mq-repo}.
Команда \textit{hg qpop} снимает патч, находящийся на вершине стека. MQ
отслеживает применённые патчи, то есть текущее состояние стека, и хранит их
список в файле \textit{status}:
\small{\begin{lstlisting}
937b07ba19ea9d98c9e3433e5ba535bc64ddecaf:wiki_external_filter_plugin.patch
7c32cb90bb7f23cfdb9d60a130cc3a11ef426814:wiki_formatting_routine.patch
\end{lstlisting}}
Каждый патч характеризуется названием, расположенным после двоеточия, и
идентификатором связанного с ним коммита в родительском репозитории.
Расширение MQ предоставляет множество команд для управления патчами, основные
из них:
\begin{itemize}
  \item \textit{qnew} создание патча;
  \item \textit{qpop} снятие патча;
  \item \textit{qpush} применение патча;
  \item \textit{qdiff} просмотр изменений, вносимых патчем;
  \item \textit{qrefresh} сохранение изменений в патч;
  \item \textit{qcommit} сохранение изменений в репозиторий патчей;
  \item \textit{qfinish} оформить патч, как коммит.
\end{itemize}

\subsection{Управление патчами}
Для того, чтобы использовать MQ был создан репозиторий внутри
установочной директории Redmine. Метаданные репозитория хранятся в папке
\textit{.hg} и не затрагиваются обновлением приложения. В репозиторий
были добавлены файлы, к которым будут применятся патчи. Затем был
инициализирован MQ репозиторий командой \textit{hq init --mq}.

Все патчи были занесены MQ репозиторий и плагины были также оформлены в виде
патчей для того, чтобы автоматизировать процесс их установки.
Автоматизированный процесс обновления приложения показан на рисунке
\ref{picture:mq-workflow}. Описать его можно следующим образом: сначала
снимаются все патчи, с помощью команды \textit{hg qpop -a}, затем приложение
обновляется и все патчи применяются с помощью команды \textit{hg qpush -a}. С
помощью данного подхода удалось отказаться от ручной установки плагинов и
применения патчей с помощью утилиты textit{patch}.

При обновлении Redmine некоторые расширения могут стать неработоспособными и
потребовать внесения изменений. Был внедрён процесс, показанный на рисунке
\ref{picture:mq-workflow}, позволяющий эффективно осуществлять поддержку
расширений при обновлениях. Процесс внесения изменений при неисправном патче
описывается следующим образом: необходимо извлечь неработающий патч на вершину
стека с помощью команды \textit{hg qpop <имя\_патча>}, внести изменения в
исходные коды приложения и по окончании записать изменения в файл патча с
помощью команды \textit{hg qrefresh}. Поскольку патчи хранятся в MQ
репозитории, то следующим шагом следует выполнить коммит командой \textit{hg
qcommit}, чтобы зафиксировать изменения.

\inspicture{mq-workflow}{Работа с Mercurial Queues}{1}
С помощью инструмента MQ была автоматизирована установка плагинов и патчей, а
также данный инструмент был успешно внедрён в процесс разработки расширений,
что позволило снизить трудозатраты на поддержку расширений при обновлениях
Redmine.

\section{Размещение расширений в открытом доступе}
Размещение расширений целесообразно, поскольку чем больше
пользователей использует приложение, тем больше можно получить отзывов, а
следовательно и больше идей, способствующих улучшению продукта.
Пользователь также может обладать навыками программирования и прислать патч,
исправляющий или добавляющий функциональность. С целью получения преимуществ от
привлечения сторонних пользователей и программистов, требуется выложить
разработанные расширения в открытый доступ, а также организовать открытый
процесс разработки.

\inspicture{opensource-feedback}{Взаимодействие с сообществом}{0.4}

\paragraph{Ограничение доступа к вики-страницам.}
Процесс взаимодействия с сообществом отображён на рисунке
\ref{picture:opensource-feedback}. В соответствии с данным процессом исходные
коды проекта были размещены на хостинге открытых проектов github \cite{github},
были составлены инструкции по установке и информация о плагине размещена
\cite{private-wiki} на сайте Redmine (см. рисунок
\ref{picture:private-wiki-on-redmine}). Хостинг github позволяет пользователям
участвовать в разработке приложения, позволяя создавать задачи, присылать патчи
и создавать ответвления, для того, чтобы параллельно разрабатывать свою версию
приложения. Вышеназванные возможности github используются в полной мере и за
время, которое плагин находился в открытом доступе было:
\begin{itemize}
  \item привлечено 7 активных пользователей;
  \item получено 11 отзывов;
  \item внесено 4 улучшения на основе полученных отзывов; 
  \item принято 3 патча;
  \item создано 1 ответвление кодовой базы для проекта ChiliProject.
\end{itemize}
Плагин был размещён в открытом доступе и при помощи возможностей хостинга
github был организован открытый процесс разработки, что позволило улучшить
качество плагина.
\inspicture{private-wiki-on-redmine}{Страница плагина на сайте Redmine}{1}

\paragraph{Ограничение доступа к репозиториям.}
К плагину была составлена документация и репозиторий с плагином был размещён на
хостинге github. На данный момент доступ к плагину закрыт, поскольку выход
плагина совпал с выходом новой версии Redmine, в следствии чего работа плагина
была нарушена. В ближайшее время в плагин будут внесены исправления и к нему
будет открыт доступ.

\paragraph{Рассылка уведомлений о задачах.}
К плагину была составлена документация и репозиторий с плагином был размещён на
хостинге github. На данный момент доступ к плагину закрыт, поскольку он
проходит тестирование в Ярославской лаборатории FRUCT. После того, как
корректность работы плагина будет проверена практикой, доступ к репозиторию
будет открыт, а информация будет размещена на сайте Redmine.

\paragraph{Механизм позиционирования календаря.}
Патч был размещён в баг-трекере официального сайта Redmine, задачей под номером
11021 \cite{calendar-bug}. В течении ближайшего времени он будет рассмотрен и
возможно принят в кодовую базу Redmine.

\paragraph{Изображения-ссылки в боковой панели.}
Плагин помещает изображения-ссылки в боковую панель без возможности настройки
изображений и ссылок через интерфейс Redmine. Планируется доработать плагин,
дав возможность пользователю производить редактирование содержания, и выложить
доработанный плагин на официальный сайт Redmine.

\paragraph{Тип задачи при составлении обзора кода.}
Разработка плагина Code Review, к которому написан данный патч, ведётся на
хостинге проектов bitbucket \cite{bitbucket}. С помощью возможностей bitbucket
разработанный патч был отправлен автору плагина. Автор принял изменения
вносимые патчем \cite{pull-request}, в кодовую базу плагина.


%%% Local Variables: 
%%% mode: latex
%%% TeX-PDF-mode: t
%%% TeX-master: "diploma"
%%% End: 
