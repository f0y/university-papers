\newpage

\chapter*{Заключение}

\addcontentsline{toc}{chapter}{Заключение}

Была разработана архитектура сетевой подсистемы, основанная на
XMPP протоколе, позволяющая выполнять совместное редактирования диаграмм связей.
Архитектура была успешно реализована в приложении HiveMind. В настоящий момент
приложение HiveMind позволяет выполнять совместное редактирование диаграмм
связей всех целевых платформах, таких как: Maemo, MeeGo, GNU/Linux, Windows.

Приложение HiveMind может быть загружено со страницы проекта по веб-адресу:
\emph{http://linuxlab.corp7.uniyar.ac.ru/projects/hivemind}. Его исходные коды
находятся в свободном доступе в репозитории Mercurial по веб-адресу:
\emph{http://linuxlab.corp7.uniyar.ac.ru/hgpub/hivemind}.

Сетевая подсистема оперирует в терминах абстрактных комманд, что делает её
слабо связанной с предметной областью приложения. Поэтому становится возможным
применение разработанной архитектуры в других приложениях, связанных с
совместным редактированием документов.

Результаты работы докладывались на девятой конференции FRUCT в городе
Петрозаводск и опубликованы в трудах этой конференции \cite{hivemind-9th-fruct}
