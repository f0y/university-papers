\newpage

\chapter*{Заключение}

\addcontentsline{toc}{chapter}{Заключение}

Сейчас рынок мобильных устройств стремительно меняется, набирают популярность
такие его направления, как интернет-планшеты, мощные коммуникаторы и нетбуки.
Естественно, новые гаджеты требуют совершенно другого программного обеспечения.

Проект HiveMind разрабатывается группой студентов в рамках программы
русско-финского сотрудничества в области коммуникаций (FRUCT) и направлен на
совместное использование таких технологий как диаграммы связей и мобильные
устройства, для увеличения производительности труда и повышению потенциального
числа идей для решения задач в различных сферах жизни человека.

Данная работа посвящена разработке основных частей пользовательского интерфейса
редактора диаграмм связей HiveMind. Были решены следующие задачи:
\begin{itemize}
\item
организовано взаимодействие классов модели и классов вида;
\item спроектировано и разработано эргономичное контекстное меню;
\item разработан эргономичный интерфейс для редактирования текста узлов;
\item спроектирована и разработана система загрузки и выбора пиктограмм.
\end{itemize}

В дальнейшем планируется разработать интерфейс для изменения настроек
приложения, диалог редактирования других атрибутов узлов, а так же разработать
концепцию эффективного совместного редактирования диаграмм связей.
