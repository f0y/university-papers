\chapters{Заключение}

В работе была освещена разработка расширений для системы управления проектами
Redmine.

В первой части были реализованы расширения Redmine, предоставляющие
следующую функциональность:
\begin{itemize}
  \item ограничение доступа к репозиториям;
  \item ограничение доступа к отдельным вики-страницам;
  \item рассылка уведомлений о задачах; 
  \item выбор типа задачи при составлении обзора кода;
  \item механизм позиционирования календаря;
  \item изображения-ссылки в боковой панели.
\end{itemize}

Во второй части был усовершенствован процесс, обеспечивающий поддержку
расширений. Управление данным процессом с помощью инструмента Mercurial Queues,
позволило значительно снизить трудозатраты на поддержку расширений при
обновлениях Redmine.

В третьей части была проведена работа по размещению расширений в открытом
доступе. Благодаря чему, от пользователей были получены отзывы и улучшения,
которые повысили качество расширений.

Разработанные расширения используются в Ярославской лаборатории FRUCT.
Внедрение данных расширений, повысило эффективность и удобство работы
участников лаборатории.

%%% Local Variables: 
%%% mode: latex
%%% TeX-PDF-mode: t
%%% TeX-master: "diploma"
%%% End: 
