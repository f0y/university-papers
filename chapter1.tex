\chapter{Общая характеристика предметной области и постановка задачи}
\section{Системы управления проектом}
Системы управление проектами являются комплексными системами, которые
предоставляют большой спектр функциональных возможностей для
деятельностей, связанных с разработкой проектов. Функционал, обеспечивающий
следующие виды деятельности, присутствует во многих системах управления
проектами:
\begin{itemize}
  \item Планирование;
  \item Постановка задач;
  \item Оценка и разработка;
  \item Управление бюджетом;
  \item Распределение ресурсов;
  \item Контроль качества;
  \item Взаимодействие;
  \item Документирование.
\end{itemize}
% Взаимодействие, Требования, Тесты, Задачи, Дефекты, Документы, Изменения,
% Время, Кастомизация, Ресурсы, Отчеты, Безопасность
Системы управления проектами позволяют сделать процесс разработки более
эффективным и удобным, таким образом противодействуя высокой сложности и
большому количеству разрабатываемых проектов. 

\subsection{Назначение}

Одна из основных возможностей систем управления проектами "--- это
управление множеством событий и задач. Основные действия, производимые
при управлении задачами, следующие:
\begin{itemize}
  \item Разрешение зависимостей между задачами;
  \item Распределение людей и ресурсов на задачи;
  \item Решение проблем, связанных с неточной оценкой срока выполнения задач.
\end{itemize}

При помощи систем управления проектами возможно выполнять спектр задач
связанных с планированием, поскольку они предоставляют участникам проекта
информацию, необходимую для измерения текущих затрат на проект, и помогают
оценить объем работ и сроки, необходимые для завершения проекта. Следующая
информация может быть предоставлена системой управления проектами:
\begin{itemize}
  \item Ожидаемое время выполнения задач;
  \item Предупреждения о возможных рисках в проекте;
  \item Объём выполняемых работ;
  \item Прогресс проекта (настоящие и ожидаемые результаты);
  \item Стоимость разработки.
\end{itemize}


\subsection{Основные представители}
На рынке приложений существует огромное количество систем для управления
проектами. Здесь будут рассмотрены только бесплатные веб-ориентированные
приложения являющиеся Open Source проектами.

\subsubsection{Collabtive}
Collabtive "--- это Open Source альтернатива для коммерческих приложений, таки
как: Basecamp или ActiveCollab. Позволяет совместно участвовать в
проектах, обмениваясь мгновенными сообщениями, производить учёт времени,
строить отчеты и импортировать данные из BaseCamp.

\subsubsection{Project HQ}
Project HQ "--- это инструмент для управления проектами, написанный на Python и
использующий следующие технологии: SQLAlchemy, Pylons. Предоставляет
функциональность схожую с Collabtive, дополнительно позволяет настраивать стили
отображения с помощью CSS.

\subsubsection{Trac}
Trac "--- это инструмент для управления проектами и отслеживания ошибок,
ориентированный на разработчиков приложений. Он позволяет связать информацию
между системами контроля версий, задачи и вики-страницами. Также может служить
веб-интерфейсом к следующим системам контроля версий: Subversion, Git,
Mercurial, Bazaar and Darcs.

\subsubsection{Redmine}
Redmine "--- это гибкая система управления проектами, написанная c
использованием Ruby on Rails и являющаяся кросс-платформенной системой
поддерживающей три базы данных. Redmine предоставляет доступ к календарю и
диаграмме ганта для отображения визуальной информации о проектах и сроках,
связанных с ними.

\subsection{Соответствие функционала запросам организации}
% What is an IT Project?
% 
% Like any project, an IT project is a temporary endeavor (with a start date and
% an end date) to bring about a specific finalized goal. Here are several
% examples of IT projects:
% 
% Programming computer software, a mobile app, or video game Designing hardware
% architecture for a computer platform Web development for an online shopping
% site Data security on a social network or bank server Today, because
% information technology is such a fast-growing industry, even projects that are
% not exactly defined as “IT” (such as those in the construction or services
% industries) are not entirely separate from IT. For instance, a concert is not
% an IT project, but the featured band might advertise the event by creating a
% new website.
% 
% So, What is IT Project Management?
% 
% IT project management is the knowledge base in which IT project managers refer
% to in order to successfully carry out their projects. IT project management
% consists of the various methodologies and tools that assist in the planning,
% moderation, and execution of an IT project. The project manager is in charge of
% gathering, organizing, and directing the resources necessary to provide a
% project with the most efficient result. Because IT projects rely heavily on
% data management, one of the best things an IT project manager can do to
% increase his or her productivity is utilize the most up-to-date IT project
% management software solutions.
% 
% What is IT Project Management Software?
% 
% IT project management software is a tool that enhances the capabilities of the
% project manager by providing more accurate views of the project through its
% various stages. Because being able to make sense of large amounts of data is
% crucial to the success of an IT project, IT project management software allows
% a manager to have increased control over what information is visible.
% 
% Today, more IT project management solutions are moving to SaaS
% (software-as-a-service), making the project management processes more optimized
% to the needs of the project manager. Because many of these tools are web-based,
% IT project managers and individuals alike are able to centralize their data in
% one location, allowing quick communication and collaboration – aspects that are
% essential in the fast-paced, ever-changing world of IT.

\section{Redmine}
% Project management software can be implemented as a Web application, accessed
% through an intranet, or an extranet using a web browser.
%   This has all the usual advantages and disadvantages of web applications:
%   Can be accessed from any type of computer without installing software on
%   user's computer.
%   Ease of access-control.
%   Naturally multi-user.
%   Only one software version and installation to maintain.
%   Centralized data repository.
%   Typically slower to respond than desktop applications.
%   Project information not available when the user (or server) is offline.
%   Some solutions allow the user to go offline with a copy of the data.


\subsection{Функциональные возможности}
% Some of the main features of Redmine are:
% 
% Multiple projects support
% Flexible role based access control
% Flexible issue tracking system
% Gantt chart and calendar
% News, documents and files management
% Feeds and email notifications
% Per project wiki
% Per project forums
% Time tracking
% Custom fields for issues, time-entries, projects and users
% SCM integration (SVN, CVS, Git, Mercurial, Bazaar and Darcs)
% Issue creation via email
% Multiple LDAP authentication support
% User self-registration support
% Multilanguage support
% Multiple databases support

\subsection{Возможности для расширения}



\section{Постановка задачи}
% (аспекты: функционал, обеспечение возможности поддержки, возврат наработок
% сообществу)



%%% Local Variables: 
%%% mode: latex
%%% TeX-PDF-mode: t
%%% TeX-master: "diploma"
%%% End: 
