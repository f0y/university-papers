\chapter{Общая характеристика предметной области и постановка задачи}
\section{Системы управления проектом}
Системы управление проектами являются комплексными системами, которые
предоставляют большой спектр функциональных возможностей для
деятельностей, связанных с разработкой проектов. Функционал, обеспечивающий
следующие виды деятельности, присутствует во многих системах управления
проектами:
\begin{itemize}
  \item Планирование;
  \item Постановка задач;
  \item Оценка и разработка;
  \item Управление бюджетом;
  \item Распределение ресурсов;
  \item Контроль качества;
  \item Взаимодействие;
  \item Документирование.
\end{itemize}
% Взаимодействие, Требования, Тесты, Задачи, Дефекты, Документы, Изменения,
% Время, Кастомизация, Ресурсы, Отчеты, Безопасность
Системы управления проектами позволяют сделать процесс разработки более
эффективным и удобным, таким образом противодействуя высокой сложности и
большому количеству разрабатываемых проектов. 

\subsection{Назначение}

Одна из основных возможностей систем управления проектами "--- это
управление множеством событий и задач. Основные действия, производимые
при управлении задачами, следующие:
\begin{itemize}
  \item Разрешение зависимостей между задачами;
  \item Распределение людей и ресурсов на задачи;
  \item Решение проблем, связанных с неточной оценкой срока выполнения задач.
\end{itemize}

При помощи систем управления проектами возможно выполнять спектр задач
связанных с планированием, поскольку они предоставляют участникам проекта
информацию, необходимую для измерения текущих затрат на проект, и помогают
оценить объем работ и сроки, необходимые для завершения проекта. Следующая
информация может быть предоставлена системой управления проектами:
\begin{itemize}
  \item Ожидаемое время выполнения задач;
  \item Предупреждения о возможных рисках в проекте;
  \item Объём выполняемых работ;
  \item Прогресс проекта (настоящие и ожидаемые результаты);
  \item Стоимость разработки.
\end{itemize}


\subsection{Основные представители}
На рынке приложений существует огромное количество систем для управления
проектами. Здесь будут рассмотрены только бесплатные веб-ориентированные
приложения являющиеся Open Source проектами.

\subsubsection{Collabtive}
Collabtive "--- это Open Source альтернатива для коммерческих приложений, таких
как: Basecamp или ActiveCollab. Позволяет совместно участвовать в
проектах, обмениваясь мгновенными сообщениями, производить учёт времени,
строить отчеты и импортировать данные из BaseCamp.

\subsubsection{Project HQ}
Project HQ "--- это инструмент для управления проектами, написанный на Python и
использующий следующие технологии: SQLAlchemy, Pylons. Предоставляет
функциональность схожую с Collabtive, дополнительно позволяет настраивать стили
отображения с помощью CSS.

\subsubsection{Trac}
Trac "--- это инструмент для управления проектами и отслеживания ошибок,
ориентированный на разработчиков приложений. Он позволяет связать информацию
между системами контроля версий, задачами и вики-страницами. Также может
служить веб-интерфейсом к следующим системам контроля версий: Subversion, Git,
Mercurial, Bazaar and Darcs.

\subsubsection{Redmine}
Redmine "--- это гибкая система управления проектами, написанная c
использованием Ruby on Rails и являющаяся кросс-платформенной системой
поддерживающей три базы данных. Redmine предоставляет доступ к календарю и
диаграмме ганта для отображения визуальной информации о проектах и сроках,
связанных с ними.

\subsection{Соответствие функционала запросам организации}
Ярославская лаборатория `Fruct` занимается разработкой проектов в рамках
одноимённой программы. Проектами лаборатории являются мобильные приложения,
написанные с использованием фреймворка Qt и предназначенные для запуска на
различных платформах: MeeGo, Symbian, Android.

Мир информационных технологий чрезвычайно быстро развивается и изменяется.
Таким образом менеджеры и участники проектов постоянно сталкиваются с
изменением требований бизнеса или заказчика, к которым они должны
адаптироваться, а также вследствии специфики информационных областей приходится
обрабатывать большое количество данных. Поэтому управление проектами в
информационных областях является сложным процессом, облегчить выполнение
которого могут системы управления проектами.

Система управления проектами позволяет увеличить производительность участников
проекта, из чего следует, что её использование является оптимальным решением.
Но при её выборе необходимо учесть тот факт, что система, помимо наличия
достаточных функциональных возможностей, должна быть легко расширяемой.
Мир информационных технологий быстро изменяется, поэтому система управления
информационными проектами, так же должна иметь способность к расширению и
изменению. Миграция на другую систему управления проектами, удовлетворяющую
новым требованиям, в большинстве случаев невозможно из-за разницы в
представлении данных и сложных взаимосвязей внутри них. Поэтому одним из важных
требований к системам управления информационными проектами является возможность
расширения.

\section{Redmine}
% Project management software can be implemented as a Web application, accessed
% through an intranet, or an extranet using a web browser.
%   This has all the usual advantages and disadvantages of web applications:
%   Can be accessed from any type of computer without installing software on
%   user's computer.
%   Ease of access-control.
%   Naturally multi-user.
%   Only one software version and installation to maintain.
%   Centralized data repository.
%   Typically slower to respond than desktop applications.
%   Project information not available when the user (or server) is offline.
%   Some solutions allow the user to go offline with a copy of the data.


\subsection{Функциональные возможности}
% Some of the main features of Redmine are:
% 
% Multiple projects support
% Flexible role based access control
% Flexible issue tracking system
% Gantt chart and calendar
% News, documents and files management
% Feeds and email notifications
% Per project wiki
% Per project forums
% Time tracking
% Custom fields for issues, time-entries, projects and users
% SCM integration (SVN, CVS, Git, Mercurial, Bazaar and Darcs)
% Issue creation via email
% Multiple LDAP authentication support
% User self-registration support
% Multilanguage support
% Multiple databases support

\subsection{Возможности для расширения}



\section{Постановка задачи}
% (аспекты: функционал, обеспечение возможности поддержки, возврат наработок
% сообществу)



%%% Local Variables: 
%%% mode: latex
%%% TeX-PDF-mode: t
%%% TeX-master: "diploma"
%%% End: 
