\chapter{Общая характеристика предметной области и постановка задачи}
\section{Системы управления проектом}
Системы управление проектами являются комплексными системами, которые
предоставляют большой спектр функциональных возможностей для деятельностей,
связанных с разработкой проектов. Функционал, обеспечивающий следующие виды
деятельности, присутствует во многих системах управления проектами:
\begin{itemize}
  \item Планирование;
  \item Постановка задач;
  \item Оценка и разработка;
  \item Управление бюджетом;
  \item Распределение ресурсов;
  \item Контроль качества;
  \item Взаимодействие;
  \item Документирование.
\end{itemize}
Системы управления проектами позволяют сделать процесс разработки более
эффективным и удобным, таким образом противодействуя высокой сложности и
большому количеству разрабатываемых проектов. 

\subsection{Назначение}

Одна из основных возможностей систем управления проектами "--- это
управление множеством событий и задач. Основные действия, производимые
при управлении задачами, следующие:
\begin{itemize}
  \item Разрешение зависимостей между задачами;
  \item Распределение людей и ресурсов на задачи;
  \item Решение проблем, связанных с неточной оценкой срока выполнения задач.
\end{itemize}

При помощи систем управления проектами возможно выполнять спектр задач
связанных с планированием, поскольку они предоставляют участникам проекта
информацию, необходимую для измерения текущих затрат на проект, и помогают
оценить объем работ и сроки, необходимые для завершения проекта. Следующая
информация может быть предоставлена системой управления проектами:
\begin{itemize}
  \item Ожидаемое время выполнения задач;
  \item Предупреждения о возможных рисках в проекте;
  \item Объём выполняемых работ;
  \item Прогресс проекта (настоящие и ожидаемые результаты);
  \item Стоимость разработки.
\end{itemize}


\subsection{Основные представители}
На рынке приложений существует огромное количество систем для управления
проектами. Здесь будут рассмотрены только бесплатные веб-приложения
приложения являющиеся Open Source проектами.

\subsubsection{Collabtive}
Collabtive "--- это Open Source альтернатива для коммерческих приложений, таких
как: Basecamp или ActiveCollab. Позволяет совместно участвовать в
проектах, обмениваясь мгновенными сообщениями, производить учёт времени,
строить отчеты и импортировать данные из BaseCamp.

\subsubsection{Project HQ}
Project HQ "--- это инструмент для управления проектами, написанный на Python и
использующий следующие технологии: SQLAlchemy, Pylons. Предоставляет
функциональность схожую с Collabtive, дополнительно позволяет настраивать стили
отображения с помощью CSS.

\subsubsection{Trac}
Trac "--- это инструмент для управления проектами и отслеживания ошибок,
ориентированный на разработчиков приложений. Он позволяет связать информацию
между системами контроля версий, задачами и вики-страницами. Также может
служить веб-интерфейсом к следующим системам контроля версий: Subversion, Git,
Mercurial, Bazaar and Darcs.

\subsubsection{Redmine}
Redmine "--- это гибкая система управления проектами, написанная c
использованием Ruby on Rails и являющаяся кросс-платформенной системой
поддерживающей три базы данных. Redmine предоставляет доступ к календарю и
диаграмме Ганта для отображения визуальной информации о проектах и сроках,
связанных с ними.

\subsection{Соответствие функционала запросам организации}
Ярославская лаборатория `Fruct` занимается разработкой проектов в рамках
одноимённой программы. Проектами лаборатории являются мобильные приложения,
написанные с использованием фреймворка Qt и предназначенные для запуска на
различных платформах: MeeGo, Symbian, Android.

Мир информационных технологий чрезвычайно быстро развивается и изменяется.
Таким образом менеджеры и участники проектов постоянно сталкиваются с
изменением требований бизнеса или заказчика, к которым они должны
адаптироваться, а также вследствие специфики информационных областей
приходится обрабатывать большое количество данных. Поэтому управление проектами в
информационных областях является сложным процессом, облегчить выполнение
которого могут системы управления проектами.

При выборе системы управления проектами необходимо учесть тот факт, что, помимо
наличия достаточных функциональных возможностей, должна быть легко расширяемой.
Мир информационных технологий быстро изменяется, поэтому система управления
информационными проектами, так же должна иметь способность к расширению и
изменению. Миграция на другую систему управления проектами, удовлетворяющую
новым требованиям, в большинстве случаев невозможно из-за разницы в
представлении данных и сложных взаимосвязей внутри них. Поэтому одним из важных
требований к системам управления информационными проектами является возможность
расширения.

\section{Redmine}
Redmine "--- это система управления проектами, основными достоинствами которой
являются, бесплатность, гибкая расширяемость, доступные исходные коды.
Данная система позволяет управлять всеми стадиями разработки проекта, начиная
от обсуждения проектного плана и заканчивая отслеживанием задач и занесением
данных об их прогрессе в общую базу знаний проекта. Одной из главных
особенностей Redmine является большое сообщество разработчиков, сложившееся
вокруг него. Redmine "--- это Open Source проект и каждый желающий может принять
участие в его разработке, тем самым способствуя развитию проекта.

\subsection{Функциональные возможности}
Redmine обладает следующими возможностями:
\begin{itemize}
  \item Неограниченное количество проектов;
  \item Гибкая система контроля доступа;
  \item Гибкая система отслеживания задач;
  \item Диаграмма Ганта и календарь;
  \item Новости, документы и управление файлами;
  \item RSS и уведомления по e-mail;
  \item Вики страницы;
  \item Форумы;
  \item Отслеживание времени;
  \item Настраиваемые поля для задач, проектов и пользователей;
  \item Интеграция с системами контроля версий;
  \item Создание задач посредством e-mail;
  \item Аутентификация через LDAP;
  \item Регистрация инициируемая пользователем;
  \item Поддержка множества языков;
  \item Поддержка нескольких БД.
\end{itemize}
Проект активно развивается и с выходом новых версии список возможностей
расширяется всё больше.

\subsection{Возможности для расширения}
Существует два способа расширить функциональность Redmine: непосредственное
изменение исходного кода и оформление его в виде патчей и разработка
обособленных модулей к системе, называемых плагинами.

\subsubsection{Патч}
Изменение функциональности с помощью патча является универсальным способом,
применимым к любому Open Source приложению. Патч представляет собой файл, в
котором отражена разница между оригинальными и изменёнными файлами. Патч в
применяется к оригинальным файлам, чтобы получить их модифицированные версии. С
помощью патчей мы получаем возможность удобным образом распространять изменения
исходного кода приложения.

\subsubsection{Плагин}
Плагин "--- это модуль, который подключается к приложению, с целью изменения
или добавления новой функциональности. Система плагинов Redmine основана на
механизме Rails Engines и специальным образом расширена разработчиками Redmine.
Плагины являются очень популярным (создано более 300 плагинов) и удобным,
для установки пользователем, средством расширения Redmine. Но в тоже время они
достаточно сложны в разработке и ограничены в возможностях.

\subsubsection{Сравнение способов расширения}
У каждого из способов есть свои достоинства и недостатки, которые приведены в
таблице \ref{comparing_extensions}.
\begin{table}[hb!]
\small
\centering
\begin{tabular}{ 
|>{\centering\arraybackslash}m{0.4\textwidth}
|>{\centering\arraybackslash}m{0.25\textwidth}
|>{\centering\arraybackslash}m{0.25\textwidth}|}
\hline
\textbf{Сравниваемый параметр} & \textbf{Плагин} & \textbf{Патч}\\
\hline
Сложность разработки & Высокая & Низкая \\
\hline
Спектр решаемых задач & Ограниченный & Максимальный\\
\hline
Устойчивость к обновлениям системы & Высокая & Низкая \\
\hline
Стоимость внесения изменений & Высокая & Средняя \\
\hline
Удобство распространения & Высокое & Низкое \\
\hline
\end{tabular}
\caption{Сравнение способов расширения Redmine}
\label{comparing_extensions}
\end{table}


\section{Постановка задачи}
Redmine в лаборатории `FRUCT` используется не только для внутренних нужд, но
также позволяет внешним пользователям следить за развитием общедоступных
проектов. Redmine обладает гибкой системой контроля прав доступа, которая тем
не менее не в полной мере отвечает потребностям лаборатории. Она не даёт
возможность контролировать доступ к репозиториям на уровне проектов и также нет
контроля доступа к отдельным вики-страницам. Что не позволяет создавать
частично закрытые проекты, к примеру проекты, у которых лишь часть вики-страниц
видна внешним пользователям. С связи с этим, необходимо внести следующую
функциональность:

\begin{itemize}
  \item Возможность ограничения доступа к репозиториям на уровне отдельных
  проектов;
  \item Возможность ограничения доступа к отдельным вики-страницам;
  \item Изображения-ссылки на рекомендуемые ресурсы в боковой панели.
\end{itemize} 
   
Всё взаимодействие участников проекта проходит через Redmine, по этой причине
важно, чтобы работа с Redmine была эффективной и удобной. Следующие изменения,
должны быть внесены, для того чтобы повысить удобство использования системой:
\begin{itemize}
  \item Реализовать рассылку уведомлений о приближающихся и просроченных
  задачах;
  \item Расширить интеграцию с системами контроля версий; 
  \item Улучшить механизм позиционирования всплывающего календаря;
  \item Улучшить систему навигации между страницами просмотра изменений.
\end{itemize}

Разработку расширений целесообразно производить в итеративном стиле, постепенно
увеличивая функциональность и исправляя дефекты. Также будет необходимо
осуществлять поддержку, поскольку Redmine регулярно обновляется и возможно,
что одно из подобных обновлений сделает некоторые расширения
неработоспособными и они потребуют новых исправлений. Исходя из этого,
необходимо разработать механизм позволяющий эффективно управлять 
итеративной разработкой и поддержкой большого количества расширений.

Redmine "--- это Open Source проект, который развивается разработчиками по
всему миру. В интересах лаборатории способствовать тому, чтобы популярность
Redmine росла, поскольку количество энтузиастов, участвующих в его разработке,
напрямую зависит от популярности проекта. Необходимо выложить максимальное
количество разработанных расширений в открытый доступ, чтобы еще более
расширить его функциональность в глазах новых пользователей. Возврат наработок
также может привлечь сторонних разработчиков, заинтересовавшихся
функциональностью выложенных расширений.



%%% Local Variables: 
%%% mode: latex
%%% TeX-PDF-mode: t
%%% TeX-master: "diploma"
%%% End: 
