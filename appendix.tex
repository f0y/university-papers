\appendix

\makeatletter
\gdef\thechapter{\@Asbuk\c@chapter}
\makeatother

\chapter{Глоссарий}
\label{glossary}
Репозиторий "--- это место в файловой системе, управляемое системой контроля
версий, в которой хранятся исходные коды приложения и различные документы
вместе с историей их изменения и другой служебной информацией.

Вики-страница "--- это страница на сайте, содержимое которой пользователи могут
самостоятельно изменять с помощью инструментов предоставляемых сайтом.
Форматирование текста на таких страницах производится использованием
вики-разметки.

Метапрограммирование "--- это процесс написание компьютерных программ,
способных создавать и изменять другие программы или самих себя на этапе
компиляции или во время выполнения.

Ruby "--- динамический язык, который предоставляет широкие возможности для
метапрограммирования.

Модуль "--- Часть кода, выделенная специальным образом и
содержащая набор методов и констант.

Хук (Hook)"--- это программный интерфейс, который позволяет расширить
функциональность Redmine. При помощи Hook'a автор плагина имеет возможность
зарегистрировать функции обратного вызова, которые будут вызваны одна за
другой, при достижении участка кода, в котором расположен Hook.

Helper "--- это вспомогательный метод, позволяющий выделить
повторяющийся код в методы и получить преимущества от его многократного
использования.

Дерево DOM "---  это не зависящий от платформы и языка программный интерфейс,
позволяющий программам и скриптам получить доступ к содержимому HTML, XHTML и
XML-документов, а также изменять содержимое, структуру и оформление таких
документов.


\chapter{Исходные тексты расширений}

\section{Патч ensure-that-calendar-is-visible.patch}
\label{appendix:ensure-that-calendar-is-visible.patch}
\lstset{language={diff}}
\begin{lstlisting}
diff --git a/public/javascripts/calendar/calendar-setup.js b/public/javascripts/calendar/calendar-setup.js
--- a/public/javascripts/calendar/calendar-setup.js
+++ b/public/javascripts/calendar/calendar-setup.js
@@ -193,6 +193,18 @@
      cal.showAtElement(params.button || params.displayArea || params.inputField);
    else
      cal.showAt(params.position[0], params.position[1]);
+
+        var elementOffsets = $(cal.element).cumulativeOffset();
+        new Effect.Parallel(
+        [
+            new Effect.Tween(null, document.viewport.getScrollOffsets().top, elementOffsets[1], {sync: true},
+                function(p){ scrollTo(document.viewport.getScrollOffsets().left, p.round());}),
+            new Effect.Tween(null, document.viewport.getScrollOffsets().left, elementOffsets[0], {sync: true},
+                function(p){ scrollTo(p.round(), document.viewport.getScrollOffsets().top);})
+        ],
+        {duration: 1}
+        );
+
    return false;
  };
\end{lstlisting}

%%% Local Variables: 
%%% mode: latex
%%% TeX-PDF-mode: t
%%% TeX-master: "diploma"
%%% End: 
