\appendix

\makeatletter
\gdef\thechapter{\@Asbuk\c@chapter}
\makeatother

\chapter{Глоссарий}
\label{glossary}
Расширение "--- это файл или набор файлов, содержащих программный код и
позволяющих расширить или изменить поведение программной системы.

Репозиторий "--- это место в файловой системе, управляемое системой контроля
версий, в которой хранятся исходные коды приложения и различные документы,
вместе с историей их изменения и другой служебной информацией.

Вики-страница "--- это страница на сайте, содержимое которой пользователи могут
самостоятельно изменять с помощью инструментов предоставляемых сайтом.
Форматирование текста на таких страницах производится использованием
специального языка разметки.

Метапрограммирование "--- это процесс написание компьютерных программ,
способных создавать и изменять другие программы или самих себя на этапе
компиляции или во время выполнения.

Ruby "--- динамический язык, который предоставляет широкие возможности для
метапрограммирования.

Модуль "--- Часть кода, выделенная специальным образом и содержащая набор
методов и констант.

Хук (Hook)"--- это программный интерфейс, который позволяет расширить
функциональность Redmine. При помощи хукa автор плагина имеет возможность
зарегистрировать функции обратного вызова, которые будут вызваны одна за
другой, при достижении участка кода, в котором расположен хук.

Вспомогательный метод (Helper) "--- это  метод, позволяющий выделить
повторяющийся код в методы и получить преимущества от его многократного
использования.

Дерево DOM "---  это не зависящий от платформы и языка программный интерфейс,
позволяющий программам и скриптам получить доступ к содержимому HTML, XHTML и
XML-документов, а также изменять содержимое, структуру и оформление таких
документов.

Cron "--- это планировщик задач в UNIX-подобных операционных системах,
использующийся для периодического выполнения заданий в определённое время.
Регулярные действия описываются инструкциями, в конфигурационном файле.

Rake "--- это инструмент для автоматизации сборки программного кода.


\chapter{Исходные тексты расширений}

\section{Патч ensure-that-calendar-is-visible.patch}
\label{appendix:ensure-that-calendar-is-visible.patch}
\lstset{language={diff}}
\small{\begin{lstlisting}
@@ -193,6 +193,18 @@
      cal.showAtElement(params.button || params.displayArea || params.inputField);
    else
      cal.showAt(params.position[0], params.position[1]);
+
+        var elementOffsets = $(cal.element).cumulativeOffset();
+        new Effect.Parallel(
+        [
+            new Effect.Tween(null, document.viewport.getScrollOffsets().top, 
+                elementOffsets[1], {sync: true},
+                function(p){ scrollTo(document.viewport.getScrollOffsets().left, p.round());}),
+            new Effect.Tween(null, document.viewport.getScrollOffsets().left, 
+                elementOffsets[0], {sync: true},
+                function(p){ scrollTo(p.round(), document.viewport.getScrollOffsets().top);})
+        ],
+        {duration: 1}
+        );
+
    return false;
  };
\end{lstlisting}}

\section{Патч tracker-in-review-dialog.patch}
\label{appendix:tracker-in-review-dialog.patch}
\lstset{language={diff}}
\small{\begin{lstlisting}
@@ -65,7 +65,6 @@
       CodeReview.transaction {
         @review = CodeReview.new
         @review.issue = Issue.new
-        @review.issue.tracker_id = @setting.tracker_id
         @review.attributes = params[:review]
         @review.project_id = @project.id
         @review.issue.project_id = @project.id
         
--- a/vendor/plugins/redmine_code_review/app/views/code_review/_new_form.html.erb
+++ b/vendor/plugins/redmine_code_review/app/views/code_review/_new_form.html.erb
@@ -40,6 +40,10 @@
       <%= f.text_field :subject, :size => 70, :required => true %>
     </p>
     <p>
+      <label><b><%=h l(:label_tracker) %>:</b></label>
+      <%= select :issue, :tracker_id, @project.trackers.collect 
+      {|t| [t.name, t.id]}, :required => true, :selected => @setting.tracker_id%>
+    </p>
+    <p>
       <label><b><%=h l(:field_parent_issue)%>:</b></label>
       <%= f.text_field :parent_id, :size => 10 %>
       <% if @parent_candidate %>
\end{lstlisting}}

%%% Local Variables: 
%%% mode: latex
%%% TeX-PDF-mode: t
%%% TeX-master: "diploma"
%%% End: 
