\newpage

\chapter{Мотивация и постановка задачи}
\label{ch:chapter_1}

\section{Диаграммы связей}

Возможность сворачивать узлы дает большее удобство для редактирования больших
диаграмм, возможность сосредоточиться на какой то отдельной части диаграммы. Так
же при различном оформлении отдельных узлов (цвет текста, цвет фона, рамка
вокруг текста, применение пиктограмм), соединительных линий можно повысить более
четкое разграничение различных частей диаграммы. Облака обычно применяются для
выделения очень важных ключевых идей. В диаграммах связей могут быть
использованы гиперссылки на веб-страницы, локальные файлы или адреса электронной
почты. Так же используются графические связи, чтобы показать взаимосвязь
элементов, находящихся в различных частях диаграммы. На рис. можно видеть пример
применения некоторых вышеописанных атрибутов узлов.


\section{Совместное редактирование диаграмм связи}
\label{sec:collaborative_mindmapping}

В общем виде, процесс совместного редактирования диаграмм должен выглядеть
следующим образом. Один пользователь с установленным приложением HiveMind в
любой момент может открыть доступ к своей диаграмме связи. Другой пользователь
подключается к диаграмме связи данного человека и автоматически получает её
актуальную копию, после чего оба пользователя начинают совместное
редактирование. Для того чтобы предоставить различные сценарии
совместного взаимодействие, на сервере должна быть возможность изменять права
участников и режимы доступа карте. Далее будут перечислены возможные сценарии
совместного взаимодействия.

Пользователь создаёт диаграмму связи и начинает её редактирование, используя
мобильное устройство, ноутбук или персональный компьютер. В процессе
редактирования пользователь решает показать свою работу другу (к примеру, для
получения помощи). Для того чтобы сделать данное действие возможным, он
открывает доступ к своей диаграмме. В данном случае ноутбук или мобильное
устройство пользователя работает как сервис, доступный из локальной сети или
интернет. Другой пользователь подключается к диаграмме связи и автоматически
получает актуальную версию диаграммы. Далее оба пользователя начинают совместное
редактирование диаграммы. Количество взаимодействующих участников неограниченно,
поэтому другие пользователи также могут участвовать в совместной работе.
Изменения, совершаемые каждым из участников, незамедлительно посылаются
остальным участникам.

Следующий сценарий использования совместного редактирования диаграмм связи может
быть полезен при представлении докладов. К примеру, докладчик имеет
дополнительные материалы, такие как: план презентации, краткий обзор речи
выстпуление или ссылки на дополнительные ресурсы. Докладчик оформляет данные
материалы в виде диаграммы связи и открывает к ней доступ. Используя HiveMind
cлушатели подключаются к диаграмме связи, выложенной докладчиком, используя свои
мобильные устройства или нетбуки. Таким образом слушатели получают материалы
подготовленные докладчиком, а также все изменения, сделанные докладчиком в
процессе выстпуления и позволяющие более ясно предоставить материал доклада.

Диаграмма связи также может содержать материалы для дискуссии в реальном
времени. В данном случае она может быть использована как лекционная доска с
иерархической структурой, на которую каждый участник может заносить свои идеи,
комментарии или мнения. Главное преимущество такого рода обсуждений с
исполльзованием HiveMind то, что все данные имеют иерархическую структуру и
доставляются всем участникам почти мгновенно.

Пользователь может хранит на диаграмме связи свою личную информацию, поэтому,
открывая доступ к диаграмме связи, он должен иметь возможность ограничить
возможность присоединения к диаграмме только избранному кругу лиц. Также он
должен иметь возможность контролировать права доступа участников на
редактирование. К примеру, при выступлении двух докладчиков перед аудиторией,
необходимо разрешить редактирование диаграммы только выстпуающим и
запретить всем остальным.

\section{Описание проекта}\label{sec:project_summary}

\section{Обзор платформы Maemo 5}\label{sec:compare_platforms}

\section{Используемый инструментарий}\label{sec:choose_toolkit}

\section{Постановка задачи}\label{sec:statement_task}
