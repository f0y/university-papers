\newpage

\chapter{Мотивация и постановка задачи}\label{ch:chapter_1}

\section{Диаграммы связей}

Возможность сворачивать узлы дает большее удобство для редактирования больших
диаграмм, возможность сосредоточиться на какой то отдельной части диаграммы. Так
же при различном оформлении отдельных узлов (цвет текста, цвет фона, рамка
вокруг текста, применение пиктограмм), соединительных линий можно повысить более
четкое разграничение различных частей диаграммы. Облака обычно применяются для
выделения очень важных ключевых идей. В диаграммах связей могут быть
использованы гиперссылки на веб-страницы, локальные файлы или адреса электронной
почты. Так же используются графические связи, чтобы показать взаимосвязь
элементов, находящихся в различных частях диаграммы. На рис. можно видеть пример
применения некоторых вышеописанных атрибутов узлов.


\section{Use Cases For Collaborative Mind Mapping}
\label{collaboration-consept}

To allow starting teamwork at any moment we needed to choose a message exchange
system. We selected Extensible Messaging and Presence Protocol (XMPP), because
it helped us to manage with several network-related issues
\cite{hivemind-8th-fruct}.

In general, the collaboration process can be described in the following way. One
of participants publishes his/her mind map as a network service and goes on
editing it. Another user connects to the service and retrieves the latest copy
of the published mind map. After that, both users edit the map together. By
changing a policy on the server side, it is possible to support for various
teamwork scenarios.


In this chapter we present our view of collaborative work involving mind
mapping. There are some possible scenarios below.

A user creates a mind map and starts editing it using his/her mobile device,
laptop or personal computer. During mind mapping he/she decides to show his/her
work to a friend (to get some help, for example).  To do this, he/she publishes
the mind map as a network service and goes on editing. In this case the user's
device or laptop works as a server available from a local network or Internet.

Another user connects to the network service and retrieves the latest copy of
the published mind map. After that both users can edit the map together. The
number of possible participants is potentially unlimited, so other users can
also join collaboration. All the changes are sent immediately to all
participants of collaboration.

To simplify discussion of the work for all participants, mind mapping
application may provide a text messaging service. It allows users to exchange
suggestions, questions and opinions without necessity of using additional
applications.


% XXX It is very important to prevent unsynchronized changes of mind map. This
% task is solved automatically through a centralized processing of all changes.
% Server receives and processes all users changes sequentially. This approach
% excludes the following modifications before accepting of previous. Any
% participants may be a server. To do this he must have a laptop, PC or mobile
% devices having an installed application for mind mapping. There are no special
% requirements to publish a map as service.

Another use case of collaborative mind mapping relates to public speaking or
brainstorming. For example, the speaker provides additional material to the
participants, such as a plan of the presentation, detailed speech synopsis,
links to the additional resources. The other participants can see these
materials using their mobile devices/laptops. They also receive all the changes
that the author makes during his/her speech, explaining details on-the-fly.

The mind map may contain some materials for a live discussion. In this case it
can be used as a shared flip chart with the hierarchical structure for gathering
participants' ideas, comments and opinions. The advantage here is that the
results become available immediately.

At last, if the author would not like participants to edit the published
materials, he or she can prevent any modifications by using read-only mode.
Read-only mode can be also applied to a part of the mind map only.

\section{Описание проекта}\label{sec:project_summary}

\section{Обзор платформы Maemo 5}\label{sec:compare_platforms}

\section{Используемый инструментарий}\label{sec:choose_toolkit}

\section{Постановка задачи}\label{sec:statement_task}
