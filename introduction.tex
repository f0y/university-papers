\chapter{Введение}

Современный мир информационных технологий развивается чрезвычайно быстро. С каждым днём, компьютерные технологии становятся всё более доступными. За последние 10 лет в обиход прочно вошли множество компьютерных устройств, таких как: ноутбуки, нетбуки, смартфоны, планшеты. Одни из основных особенностей вышеназванных устройств - это:
\begin{itemize}
	\item Доступность для покупателя;
	\item Различные варианты использования;
	\item Многобразие установленных операционных систем; 
	\item Регулярный рост производительности. 
\end{itemize} 
Все эти факторы ведут к тому, что рынок приложений расширяется очень быстро. Появляются новые компании специализирующиеся на разработке разного рода приложений.

Разработка приложения, независимо от того, является оно десктопным, мобильным или браузерным представляет собой сложный процесс, в который вовлечены множество людей. Участники проекта взаимодействуют между собой, выполняя следующие действия: оценка, планирование, согласование, разработка, тестирование, документирование. Необходимо предоставить всем участникам проекта, начиная от разработчиков, заканчивая руководителями, интсрумент упрощающий взамодействие между ними и способствующий выполнению поставленных задач.

Системы управления проектами позволяют контролировать все аспекты связанные с разработкой проектов. Основная задача систем управления проектами, заключается в том, чтобы уменьшать сложность ведения крупных проектов и способствовать эффективному управлению несколькими проектами. В качестве системы управления проектами был выбран Redmine, как один из наиболее популярных и функциональных продуктов на рынке.

Система управления проектами Redmine -- это OpenSource веб-приложение, разрабатываемое разработчиками по всему миру. Из его преимуществ стоит отметить бесплатность, гибкую расширяемость и доступные исходные коды. Эти достоинства позволяют производить его доработку в соответсвии с нуждами организации.

В данной работе будет освещена разработка расширений для Redmine. Данные расширения добавляют функциональность и вносят улучшения в существующую систему. Мотивацией для создания расширений послужила необходимость адаптации существующей системы управления проектами к процессу разработки проектов, сложившемуся внутри компании. 

Работа состоит из трёх глав. В первой главе произведён краткий обзор систем управления проектами на предмет соответствия нуждам организации. 

Во второй главе составлены требования к расширениям и описаны детали их реализации. 

В третьей главе освещён механизм, упрощающий поддержку расширений, а также приведены результаты передачи наработок в сообщество OpenSource. 

В заключении приведены общие результаты работы и дальнейшия направления деятельности.



