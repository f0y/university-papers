\chapters{Введение}
Разработка приложения, является сложным процессом, в который
вовлечены множество людей. Участники проекта взаимодействуют между собой,
выполняя следующие действия: оценку, планирование, согласование, разработку,
тестирование, документирование. Необходимо предоставить всем участникам
проекта, начиная от разработчиков, заканчивая руководителями, инструмент
упрощающий взаимодействие между ними и способствующий выполнению поставленных
задач.

Системы управления проектами позволяют контролировать все аспекты связанные с
разработкой проектов. Основная задача систем управления проектами, заключается в
том, чтобы уменьшать сложность ведения крупных проектов и способствовать
эффективному управлению несколькими проектами.

Система управления проектами Redmine "--- это Open Source приложение,
развиваемое разработчиками по всему миру. Из его преимуществ стоит
отметить бесплатность, гибкую расширяемость и доступные исходные коды. Эти
достоинства позволяют производить его доработку в соответствии с нуждами
организации.

В работе будет освещена разработка расширений для Redmine. Данные
расширения добавляют функциональность и вносят улучшения в существующую
систему. Мотивацией для создания расширений послужила необходимость адаптации
существующей системы управления проектами к процессу разработки проектов,
сложившемуся внутри компании. Ярославская лаборатория `Fruct` занимается
разработкой приложений для мобильных платформ и использует Redmine в качестве
своего основного сайта. Особенностью является то, что Redmine используется для
внутренних нужд, одновременно позволяя внешним пользователям следить за
развитием общедоступных проектов.

Работа состоит из трёх глав. В первой главе произведён обзор предметной области
и требования, предъявляемые к расширениям. Во второй главе производится
описание деталей реализации каждого из расширений. В третьей главе освещён
механизм, упрощающий поддержку расширений, а также приведены результаты
передачи наработок сообществу.

В заключении приведены результаты работы, дальнейшие перспективы развития и
продвижения разработанных расширений.

%%% Local Variables: 
%%% mode: latex
%%% TeX-PDF-mode: t
%%% TeX-master: "diploma"
%%% End: 
