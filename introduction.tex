\chapters{Введение}

\bad{Современный мир информационных технологий развивается чрезвычайно быстро. С каждым днём, компьютерные технологии становятся всё более доступными. За последние 10 лет в обиход прочно вошли множество компьютерных устройств, таких как: ноутбуки, нетбуки, смартфоны, планшеты. Одни из основных особенностей вышеназванных устройств - это:}
\begin{itemize}
	\item \bad{Доступность для покупателя;}
	\item \bad{Различные варианты использования;}
	\item \bad{Многобразие установленных операционных систем; }
	\item \bad{Регулярный рост производительности. }
\end{itemize} 
\bad{Все эти факторы ведут к тому, что рынок приложений расширяется очень
быстро. Появляются новые компании специализирующиеся на разработке разного рода
приложений.} \rmk{Это всё мусор. Сразу начинай говорить про процесс разработки приложений}

Разработка приложения, независимо от того, является оно \bad{десктопным, мобильным или браузерным} представляет собой сложный процесс, в который вовлечены множество людей. Участники проекта взаимодействуют между собой, выполняя следующие действия: оценка, планирование, согласование, разработка, тестирование, документирование. Необходимо предоставить всем участникам проекта, начиная от разработчиков, заканчивая руководителями, интсрумент упрощающий взамодействие между ними и способствующий выполнению поставленных задач.

Системы управления проектами позволяют контролировать все аспекты связанные с
разработкой проектов. Основная задача систем управления проектами, заключается в
том, чтобы уменьшать сложность ведения крупных проектов и способствовать
эффективному управлению несколькими проектами. \bad{В качестве системы
  управления проектами был выбран Redmine, как один из наиболее популярных и
  функциональных продуктов на рынке.} \rmk{Не нужно, т.~к. её не выбирали!}

Система управления проектами Redmine "--- это Open Source приложение,
\bad{разрабатываемое разработчиками} по всему миру. Из его преимуществ стоит
отметить бесплатность, гибкую расширяемость и доступные исходные коды. Эти достоинства позволяют производить его доработку в соответсвии с нуждами организации.

В данной работе будет освещена разработка расширений для Redmine. Данные
расширения добавляют функциональность и вносят улучшения в существующую
систему. Мотивацией для создания расширений послужила необходимость адаптации
существующей системы управления проектами к процессу разработки проектов,
сложившемуся внутри компании. \rmk{Вот здесь надо чуть-чуть поподробнее. Можно
  сказать и про компанию, и про лабораторию. Добавить буквально 2 предложения,
  чтобы было ясно, откуда взялись идеи доработок.}

Работа состоит из трёх глав. В первой главе произведён \bad{краткий обзор систем
  управления проектами на предмет соответствия нуждам организации.} \rmk{Этого
  там не будет. Там будет обзор предметной области и требования, которые
  предъявляются к доработкам.}

Во второй главе составлены \bad{требования к расширениям} \rmk{это в первой
  главе, а не во второй!} и описаны детали их реализации. 

В третьей главе освещён механизм, упрощающий поддержку расширений, а также приведены результаты передачи наработок сообществу. 

В заключении приведены результаты работы и \bad{дальнейшия направления деятельности}.

%%% Local Variables: 
%%% mode: latex
%%% TeX-PDF-mode: t
%%% TeX-master: "diploma"
%%% End: 
