\newpage

\chapter*{Введение}\label{chap:introduction}
\addcontentsline{toc}{chapter}{Введение}

HiveMind is a cross-platform mind map editor. Its most important feature is
collaborative mind mapping, allowing different people to edit mind maps
together regardless of their location.

In this paper we discuss the architecture of HiveMind network subsystem and
implementation of collaborative mind map editing based on the XMPP pubsub
extension protocol. Our approach operates in terms of abstract commands, which
makes it loosely coupled from the application domain. It also makes possible
to spread our teamwork architecture to a broader class of applications, which
could benefit from collaborative document editing.

Mind map is a diagram used to represent words, tasks, ideas, or other items
linked to and arranged around a central key word or idea. There is no predefined
algorithm on how to add and structure these data, which makes mind maps suitable
for such activities as study, project management, problem solving, brainstorming.

These kinds of work often involve many people. It reveals several
issues. Firstly, all these people should be gathered in one place to begin
collaboration. Secondly, it is hard to provide equal access to all participants,
because only one person at the time can make changes to the mind map being
constructed, for instance, on a whiteboard.

HiveMind is a cross-platform collaborative mind map editor, which main idea is
to give everyone an equal opportunity to contribute to the shared mind map
regardless of their location either by using a mobile device or a personal
computer.

In general, the collaboration process can be described in the following way. One
of participants publishes his/her mind map as a network service and goes on
editing it. Another user connects to the service and retrieves the latest copy
of the published mind map. After that, both users edit the map together. By
changing a policy on the server side, it is possible to support for various
teamwork scenarios.

To allow starting teamwork at any moment we needed to choose a message exchange
system. We selected Extensible Messaging and Presence Protocol (XMPP), because
it helped us to manage with several network-related issues
\cite{hivemind-8th-fruct}.

In this paper we discuss the architecture of HiveMind network subsystem and
implementation of collaborative mind map editing based on the XMPP pubsub
extension protocol.