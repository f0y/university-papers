\newpage

\chapter*{Введение}
\label{chap:introduction}
\addcontentsline{toc}{chapter}{Введение}

В настоящее почти каждый человек, живущий в развитой стране, использует
мобильные устройства в своём быту. Существует множество классов данных
устройств, таких как: мобильные телефоны, смартфоны, КПК, планшеты. Мобильные
устройства прочно вошли в жизнь человека. С каждым годом наращивается функционал
и производительность данных устройств. Многие люди находят в данных устройствах
замену своему стационарному компьютеру. Мобильные устройства оснащаются
средствами связи, такими как wi-fi, bluetooth, 3G, что позволяет обладателю
данного устройства пользоваться интернетом.

Диаграммы связей --- это графическое представление данных, имеющих
иерархическую структуру. Они позволяют создавать, отображать и
структурировать некоторую иноформацию. У диаграмм связи нет ограничений на
структуру данных, за исключением иерархческого порядка, поэтому они могут
использоваться для работы с различной инофрмацией. Они успешно
применяются для генерации идей, конспектирования докладов, написания планов
статей, составление презентаций и так далее.

Диаграммы связи используются во множестве различных ситуаций связанных с работой
над данными. Зачастую работа над данными требует участия группы людей. В данной
ситуации возникает несколько проблем. Во-первых, людям, участвующим в данном
процессе необходимо собраться в одном месте, чтобы начать совместную работу.
Во-вторых, сложно предоставить всем участникам процесса равноценный доступ к
диаграмме связи, например только один человек может сидеть за компьютером, либо
писать на доске.

Решение вышеназванных проблем заключается в том, чтобы использовать возможности
мобильных устройств. Мобильные устройства портативны, что позволяет их с
легкостью всегда держать при себе. Также они оснащены средствами доступа в
интернет. Идея заключается в том, чтобы предоставить людям равные возможности по
внесению своего вклада в создание диаграмм, в любое время, в любом месте,
используя мобильное устройство или персональный компьютер.

В данной работе описывается реализация функции совместного редактирования в
редакторе диаграмм связей под названием HiveMind. Данный проект разрабатывается
с начала 2010 года группой студентов рамках программы FRUCT. HiveMind имеет
множество функций для редактирования диаграмм связей и поддержку платформ
Windows, Maemo, MeeGo, Linux. Проект направлен на достижение максимальной
мобильности пользователей, повышение производительности труда и увеличение
потенциальных решений в поставленных задачах.

