\chapters{Введение}
Разработка программного продукта является сложным процессом, в котором
участвуют специалисты, отвечающие за разные аспекты разработки. Участники
проекта взаимодействуют между собой, выполняя следующие действия: оценку,
планирование, согласование, разработку, тестирование, документирование.  Всем
участникам, начиная от разработчиков, заканчивая руководителями, необходим
инструмент упрощающий взаимодействие и способствующий выполнению поставленных
задач.

Системы управления проектами позволяют контролировать все аспекты связанные с
разработкой проектов. Основная задача систем управления проектами, заключается в
том, чтобы уменьшать сложность ведения крупных проектов и способствовать
эффективному управлению несколькими проектами.

Система управления проектами Redmine "--- это Open Source приложение,
развиваемое разработчиками по всему миру. Из его преимуществ стоит
отметить бесплатность, гибкую расширяемость и доступные исходные коды. Эти
достоинства позволяют производить его доработку в соответствии с нуждами
организации.

Ярославская лаборатория FRUCT ведёт исследования в области умных пространств и
применении мобильных систем в медицине и в повседневной жизни.
Также активно разрабатывает прототипы для настольных и мобильных платформ,
основанные на результатах исследований. Лаборатория использует Redmine в
качестве системы управления проектами. Взаимодействие разработчиков проходит
через данную систему и чрезвычайно важно, чтобы система предоставляла
функциональность, отвечающую нуждам лаборатории, а работа в ней была удобной и
эффективной. Система управления проектами Redmine не полностью отвечает
предъявляемым к ней требованиям, таким образом возникла задача по созданию
расширений, модифицирующих и добавляющих функционал.

Работа состоит из трёх глав. В первой главе произведён обзор предметной области
и требования, предъявляемые к расширениям. Во второй главе производится
описание деталей реализации каждого из расширений. В третьей главе освещён
механизм, упрощающий поддержку расширений, а также приведены результаты
передачи наработок сообществу. В заключении приведены разработанные расширения
и итоги взаимодействия с сообществом программистов.

%%% Local Variables: 
%%% mode: latex
%%% TeX-PDF-mode: t
%%% TeX-master: "diploma"
%%% End: 
