\newpage

\chapter*{Введение}\label{chap:introduction}
\addcontentsline{toc}{chapter}{Введение}

В современном мире мобильные устройства все больше входят в нашу жизнь. С каждым днем появляются новые модели телефонов, КПК, ноутбуков. Так же растет их производительность, функционал. Растет количество различных приложений для этих устройств. Спектр программных продуктов очень широк и приближается к многообразию программ для настольных компьютеров.

Современных мобильные устройства могут быть оснащены беспроводной передачей данных (Wi-Fi, Bluetooth), иметь достаточно большой размер памяти (16 гигабайт и выше), цветной дисплей, сенсорный дисплей, фото-видео камеру, аудио интерфейс, акселерометр.

В настоящее время мобильные телефоны, смартфоны и КПК "--- неплохая альтернатива настольным персональным компьютерам, особенно в тех случаях если нужна высокая мобильность и невысокие требования к производительности. Существуют так же и недостатки, например, физические размеры дисплея, специализированная клавиатура.

Диаграмма связей "--- это иерархическая структура, отображающая множество взаимосвязанных идей, концепций и так далее. У диаграмм связей существует много возможных применений, например, с их помощью можно разрабатывать абстрактные структуры классов, проводить мозговой штурм или планировать личные расходы, конспектировать материал, разрабатывать проекты разной сложности, проводить тренинги.

При решении сложных задач популярен метод мозгового штурма "--- оперативный метод решения проблемы на основе стимулирования творческой активности, при котором участникам обсуждения предлагают высказывать возможно большее количество вариантов решения. Затем из общего числа высказанных идей отбирают наиболее удачные, которые могут быть использованы на практике.

Обычно при составлении сложных диаграмм связей вовлечена большая группа людей разного рода занятий. При этом возникают следующие проблемы: низкая мобильность и необходимость сбора всех членов группы вместе для решения задач.

Часто при проведении мозгового штурма возникает проблемы ведения протокола обсуждения. При ведении протокола на доске или в блокноте возникает необходимость выслушивания членов группы по очереди, что в большинстве случаев неудобно. В этом случае хорошо использовать диктофон либо видеосъёмка всего процесса, что в свою очередь дает дополнительную трату ресурсов на обработку полученного материала после обсуждений.

В качестве решения проблемы совместной работы с диаграммами связей можно рассмотреть мобильные устройства. Такое решение позволяет резко повысить мобильность персонала и расширить пространство проведения обсуждений. Любой член группы может принимать участие в решении проблемы не зависимо от места его нахождения, будь он в соседней комнате или в другой стране. Так же решаются проблемы с ведением протокола обсуждений "--- каждый самостоятельно вносит варианты решений в диаграмму.

Однако при применении такого решения возникает следующие проблемы: малая распространенность данного класса приложений для мобильных устройств, а так же их скудные функциональные возможности, в том числе отсутствие поддержки совместной работы.

В данной работе описывается реализация редактора диаграмм связей под названием HiveMind. Данный проект разрабатывается с начала 2010\,г. группой студентов в рамках программы русско-финского сотрудничества в области телекоммуникаций (FRUCT). Проект направлен на достижение максимальной мобильности пользователей, повышение производительности труда и увеличения количества потенциальных решений в поставленных задачах при использовании диаграмм связей.