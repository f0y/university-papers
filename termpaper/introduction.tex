\newpage

\chapter*{Введение}
\label{chap:introduction}
\addcontentsline{toc}{chapter}{Введение}

В современном мире, почти каждый человек, живущий в развитой стране, использует
мобильные устройства в своём быту. Существует множество классов данных
устройств, таких как: мобильные телефоны, смартфоны, КПК, планшеты. Мобильные
устройства прочно вошли в жизнь человека. С каждым годом наращивается функционал
и производительность данных устройств. Многие люди находят в данных устройствах
замену своему стационарному компьютеру. Мобильные устройства оснащаются
средствами связи, такими как Wi-Fi, Bluetooth, 3G, что позволяет обладателю
данного устройства пользоваться интернетом, взаимодействуя с другими
пользователями.

Диаграммы связей --- способ изображения процесса общего системного мышления с
помощью схем, а также может рассматриваться как удобная техника альтернативной
записи. Диаграмма связей реализуется в виде древовидной схемы, на которой
изображены слова, идеи, задачи или другие понятия, связанные ветвями, отходящими
от центрального понятия или идеи. Диаграммы связей позволяют создавать,
отображать и структурировать некоторую информацию. У них нет ограничений на
структуру данных, за исключением иерархического порядка, поэтому они могут
использоваться для работы с различными данными. Диаграммы связей успешно
применяются для генерации идей, конспектирования докладов, написания планов
статей, составление презентаций и так далее.

Диаграммы связей используются во множестве различных ситуаций связанных с
работой над данными. Зачастую, работа над данными требует участия группы людей. В данной
ситуации возникает несколько проблем. Во-первых, людям, участвующим в данном
процессе, необходимо собраться в одном месте, чтобы начать совместную работу.
Во-вторых, необходимо предоставить всем участникам процесса равноценный доступ к
диаграмме связей, например только один человек может сидеть за компьютером, либо
писать на доске.

Решение вышеназванных проблем заключается в том, чтобы использовать возможности
мобильных устройств. Мобильные устройства портативны, что позволяет их с
легкостью всегда держать при себе. Также они оснащены средствами доступа в сеть
Интернет. Идея заключается в том, чтобы предоставить людям равные возможности по
внесению своего вклада в создание диаграмм, в любое время, в любом месте,
используя мобильное устройство или персональный компьютер.

В данной работе описывается реализация функций совместного редактирования в
редакторе диаграмм связей под названием HiveMind. Проект разрабатывается с
начала 2010 года группой студентов рамках программы FRUCT
\cite{hivemind-8th-fruct}. HiveMind имеет множество функций для редактирования
диаграмм связей и поддержку платформ Windows, Maemo, MeeGo, Linux. Проект
направлен на достижение максимальной мобильности пользователей, повышение
производительности труда и нахождение решений поставленных задач.
